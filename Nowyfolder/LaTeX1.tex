\documentclass{article}
\usepackage[a4paper,left=3.5cm,right=2.5cm,top=2.5cm,bottom=2.5cm]{geometry}
\usepackage[MeX]{polski}
\usepackage[cp1250]{inputenc}
%%\usepackage[utf8]{inputenc}
\usepackage[pdftex]{hyperref}
\usepackage{makeidx}
\usepackage[tableposition=top]{caption}
\usepackage{algorithmic}
\usepackage{graphicx}
\usepackage{enumerate}
\usepackage{multirow}
\usepackage{amsmath} %pakiet matematyczny
\usepackage{amssymb} %pakiet dodatkowych symboli
\title{dokument}
\author{Bartosz Aptacy}
\date{13.11.2017}
\begin{document}
\maketitle
\tableofcontents

\begin{flushleft}
Jaki� tekst
\end{flushleft}
\begin{flushright}
Jaki� tekst
\end{flushright}
\begin{center}
Jaki� tekst
\end{center}
\newpage

\begin{enumerate}%%[i)]
\item punkt 1 \$
\item punkt 2
\end{enumerate}
\begin{itemize}
\item punkt 1
\begin{itemize}
\item punkt \cite{asd}
\item punkt 2
\end{itemize}
\item punkt 2
\end{itemize}

\begin{equation}
lim _{n\rightarrow\infty}=\sum_{k-1}^{n}\frac{1}{k^{2}}=\frac{\Pi^{2}}{6}
\end{equation}
\begin{equation}
\left[x\right]_{A}=\left\{y\in U: a(x) = a(y),\forall a \in A\right\}\text{, where the central object } x\in U
\end{equation}
\begin{equation}
cos(2)=cos^{2}-sin^{2}
\end{equation}

\begin{thebibliography}{2}
\bibitem{asd}qwezxc
\bibitem{qwe}asdzxc
\end{thebibliography}
\newpage
zxc
\newpage

Tu umieszczamy kod TeXa, ktory bedzie kompilowany,?
\section{Punkt 1}
\subsection{Podpunkt 1}
\end{document}